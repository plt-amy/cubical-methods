\begin{node}
%
Homotopy type theory\marginnote{HoTT, for short.} is a relatively modern
foundational system for mathematics, which can be succinctly described
as a \emph{synthetic theory of \(\infty\)-groupoids}. In more words,
this means that the ``ambient objects'' we write mathematics in terms of
are intrinsically higher-categorical, or intrinsically
\emph{homotopical}.
%
\end{node}

\begin{node}
%
Consequently, HoTT presents a unique opportunity to do homotopy theory
\emph{on its own terms}, studying \(\infty\)-groupoids \emph{as they
are} rather than studying an \emph{encoding}, e.g. the Kan complex
encoding. This is the tradeoff of synthetic mathematics: we lose the
ability to ``peek into" the encoding, but we gain the benefit of having
our theorems automatically apply to anything which is sufficiently
\(\infty\)-groupoid\emph{-like}.
%
\end{node}

\begin{node}\label{node:hott-constructive}
%
At the same time, HoTT is\marginnote{We note that HoTT can be
consistently extended with classical principles.} a \emph{constructive}
system of mathematics, meaning that proofs are \emph{computable}, or
\emph{effective}. A constructive proof automatically doubles as an
algorithm, meaning that, for example, a proof that there are finitely
many groups of given order \emph{automatically} doubles as a procedure
for enumerating said groups.
%
\end{node}

\begin{node}
%
In HoTT, every construction we perform (and every theorem we prove) is
automatically \emph{isomorphism-invariant}, meaning that a proof of the
theorem "there are finitely many groups of order \(n\)" is automatically
a computable procedure which enumerates finite groups of order \(n\)
\emph{up to isomorphism}.
%
\end{node}

\begin{node}
%
The mathematical principle that makes HoTT ``tick'' is
\emph{univalence}, which is commonly introduced in natural language as
saying that "equivalent types are identical". We prefer \emph{not} to
use this slogan, since it could be read as a skeletality condition
(saying that there are no non-trivial equivalences of types), when the
opposite is true: we have expanded the definition of \emph{identical} to
match \emph{equivalent}.

The univalence principle is often still referred to as the univalence
\emph{axiom}, and indeed it was phrased as an axiom in
\cite{UF:2013}.
\marginnote{Consequently, the formal system presented
there is sometimes called ``axiomatic HoTT.''} This is problematic
because the introduction of axioms to a type-theoretic system destroys
\emph{canonicity}, the meta-theoretic property alluded to in
\cref{node:hott-constructive}.

This means that, if we want a foundational system where proofs are
automatically \emph{effective} and automatically
\emph{isomorphism-invariant} (or automatically \emph{homotopical}), we
need a way to \emph{prove} the univalence principle, rather than
asserting it as an axiom.
%
\end{node}
