\begin{node}
%
It is instructive to introduce the notion of \emph{universes} by
comparing them to the slightly more familiar notion of \emph{subobject
classifier}. For the start of this section, let us assume excluded
middle for simplicity.
%
\end{node}

\begin{node}
%
We can consider \emph{any} function \(f : A \to B\) between sets as
being a \emph{family of sets} indexed, confusingly, by the elements of
\(B\).  The change in direction is because the set associated by the
family to each element \(x : B\) is the \emph{fibre over \(x\)}, the set
\(f^*(x) = \{ y : A \mid f(y) = x \}.\)
%
\end{node}

\begin{remark}
When talking about sets and functions, what we called a "fibre" is more
commonly called the \emph{preimage} of \(x\). We will stick with the
homotopically-inspired terminology for consistency.
\end{remark}

\begin{node}
%
We may phrase the idea of fibres in a structural way, without mention of
elements, using the idea of \emph{pullbacks}. The fibre \(f^*(x)
\subseteq A\) is equivalently the result of pulling \(f : A \to B\) back
along an element \(x : \{*\} \to B\).

\[\begin{tikzcd}
  {\{*\} \times_B A} && A \\
  \\
  \{*\} && B
  \arrow["f"', from=1-3, to=3-3]
  \arrow["x"', from=3-1, to=3-3]
  \arrow[from=1-1, to=3-1]
  \arrow["{f^*(x)}"', from=1-1, to=1-3]
  \arrow["\lrcorner"{anchor=center, pos=0.125}, draw=none, from=1-1, to=3-3]
\end{tikzcd}\]
%
\end{node}

\begin{node}
%
Now suppose that \(f : A \to B\) is the function associated with a
subset inclusion, e.g. by letting \(A\) be "the set of even naturals",
\(B\) be \(\NN\) itself, and \(f\) be the map which forgets the
even-ness of a given number.

Investigating the family associated with this map is instructive: The
fibre over \(2\) is "the set of even numbers equal to two", i.e. the
singleton set containing \(2\). But the fibre over \(3\) is "the set of
even numbers equal to three", which is uninhabited.

Generalising, we see that our specific function \(f : A \to B\) had the
property of being \emph{injective}\marginnote{In the category of sets,
being injective is the same as being a \emph{monomorphism}. This latter
property is what defines subobject classifiers in general.}, and that
any injective function, and \emph{only} the injective functions, will
have this same behaviour for its family of fibres (i.e., the fibre over
any point will have \emph{at most one} element).
%
\end{node}

\begin{node}
%
We have characterised the injective functions in terms of the behaviour
of their associated family: A natural next question to ask if there is
an object of the category of sets to serve as the \emph{codomain} for
these families: Is there some object \(X\) such that the maps \(p : B
\to X\) correspond to injective functions \(A \mono
B\)?\marginnote{Another way to observe that the fibres of a monomorphism
have \emph{at most one} element is to note that monomorphisms are closed
under pullback: If \(f : A \mono B\) is monic, then so is the map
\(\{*\} \times_B A \to \{*\}\).}

The answer is yes, and this correspondence is very familiar: If we have
a subset \(A \subseteq B\), we can associate to each element \(x : B\)
the Boolean-valued function

\begin{equation*}
  \chi(x) = \begin{cases*}
    1 & if $x \in A$ \\
    0 & otherwise,
  \end{cases*}
\end{equation*}

and if we have such a Boolean-valued function, we recover a subset
\[
\{x : A \mid \chi(x) = 1\}.
\]
%
\end{node}

\begin{node}
%
Rephrasing this structurally,\marginnote{And still assuming excluded
middle} we have found a \emph{generic injection}: We have a particular
set (the set \(\{0,1\}\) of Booleans) and an injection \(\mathrm{true} :
\{1\} \mono \{0,1\}\), such that any other injection is an "instance" of
\(\mathrm{true}\).
%
\end{node}

\begin{node}
%
More precisely, given an injection \(f : A \mono B\), we have a unique
map \(\chi_f\) such that the diagram
\[\begin{tikzcd}
  A && {\{*\}} \\
  \\
  B && {\{0,1\}}
  \arrow["{\mathrm{true}}", from=1-3, to=3-3]
  \arrow["{\chi_f}"', from=3-1, to=3-3]
  \arrow["f"', from=1-1, to=3-1]
  \arrow[from=1-1, to=1-3]
  \arrow["\lrcorner"{anchor=center, pos=0.125}, draw=none, from=1-1, to=3-3]
\end{tikzcd}\] is a pullback square, as indicated: given \emph{any}
injection, we have a specific map \(\chi_f\) such that the fibre of
\(\chi_f\) over "true" recovers the subset \(A\).
%
\end{node}

\begin{node}
%
Our focus on fibres and pullbacks so far has been instructive for
calculation but is a hindrance to generalisation: A cleaner way of
characterising the subobject classifier is to say it represents the
functor sending each object to its \emph{poset of subobjects}: We have a
natural isomorphism
\[
  \hom(-, \Omega) \cong \mathrm{Sub}(-).
\]
The reason pullbacks get involved is as part of the construction of the
functorial action of \(\mathrm{Sub}\).
%
\end{node}

\begin{node}
%
Inspired by this success, we might want to venture further: Is there a
generic map? Put better, is there an object \(U\) which represents the
functor \(x \mapsto \mathrm{Sets}/x\), i.e., classifying
\emph{arbitrary} families of sets?

Sadly, there is not, and the reason for failure is twofold: Initially,
we have the Russellian paradoxes of \emph{the set of all sets}. If we
had such an object, then it would have to classify the family \(U \to
\{*\}\), i.e., it would have to be "an element of itself". But there is
a more subtle objection: there is a \emph{category} of sets (which is
equivalently the slice \(\mathrm{Sets}/\{*\}\)), but there is a
\emph{set} of maps \(\{*\} \to U\).

Consider, for example, the object \(\{0,1\}\) as a family over
\(\{*\}\). It has an automorphism \(x \mapsto 1 - x\), which we would
like to (and have to!) classify, but the corresponding point
\(\chi_{\{0,1\}} : 1 \to U\) has no reasonable notion of automorphism:
it's a morphism in a \(1\)-category!
%
\end{node}
