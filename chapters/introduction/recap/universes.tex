\begin{node}
%
It is instructive to introduce the notion of \emph{universes} by
comparing them to the slightly more familiar notion of \emph{subobject
classifier}. For the start of this section, let us assume excluded
middle for simplicity.
%
\end{node}

\begin{node}
%
Suppose we have a set \(A\) and some injective function \(m : B \mono
A\)\marginnote{Even with excluded middle, we will stick to a structural
conception of sets.}, and consider some element \(x : 1 \to A\). We can
form the set
\[
  \{ y : B\ |\ m(y) = x \},
\]
i.e., the set of \emph{elements of \(B\) which \(m\) maps to \(x\)}.
%
\end{node}

\begin{node}
%
If we would like an arrow \(m : B \to A\) to indeed represent a subset
of \(A\), we would like \emph{being in the image of \(m\)} to be a
true-false statement: either some \(x : 1 \to A\) comes from some \(y :
B\) or it doesn't. There should be no possiblity for \(x : 1 \to A\) to
be the image of distinct \(y, z : 1 \to B\).
%
\end{node}

% \[
%   \begin{tikzcd}
%     {B\times_A1} && B \\
%     \\
%     1 && {A.}
%     \arrow["m"', hook, from=1-3, to=3-3]
%     \arrow["x", from=3-1, to=3-3]
%     \arrow[from=1-1, to=3-1]
%     \arrow[from=1-1, to=1-3]
%     \arrow["\lrcorner"{anchor=center, pos=0.125}, draw=none, from=1-1, to=3-3]
%   \end{tikzcd}
% \]

% Since monomorphisms are stable under pullback, we conclude that the
% arrow \(! : B \times_A 1 \to 1\) is a monomorphism, hence injective.
% %
% \end{node}

% \begin{node}
% %
% For any two maps \(f, g : X \to B \times_A 1\), we have \(\text{!} \circ
% f = \text{!} \circ g\) since \(1\) is terminal. But since \(! : B
% \times_A 1 \to 1\) is monic, we must have had \(f = g\) to begin with.

% This immediately implies that any two points \(f, g : 1 \to B\times_A
% 1\) are equal.
% %
% \end{node}
