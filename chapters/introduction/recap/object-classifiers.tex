\begin{node}
%
The solution is to consider our objects living not in a 1-category, but
in an \(\infty\)-category\marginnote{Note that wherever we say
\(\infty\)-category, we mean \((\infty,1)\)-category.}: there, our
points can have their own \emph{auto-2-morphisms}, and those can have
their own auto-3-morphisms, up to infinity. The \(\infty\)-categories we
are interested in will all have \emph{enough object classifiers}.
%
\end{node}

\begin{defn}\label{defn:enough-universes}
An \(\infty\)-category \(\mathcal{C}\) has \emph{enough object
classifiers}, or \emph{enough universes}, if for every regular cardinal
\(\kappa\), there is an object \(\Type_\kappa\) which classifies the
functor
\[
X \mapsto \Core_{\le \kappa}(\mathcal{C}/X),
\]
which sends each object \(X\) to (the core of) the
sub-\(\infty\)-category \(\mathcal{C}/X\) consisting of the morphisms
with \(\kappa\)-small fibres.
\end{defn}

\begin{node}
%
The definition above is external to type theory, but we can internalise
it through the fine art of handwaving\marginnote{Another way of putting
this is that \cref{defn:enough-universes} is semantic, but we'd like a
\emph{syntactic} criterion}: We can consider the objects of the
category, up to handwaving, to be our "types". Then
\cref{defn:enough-universes} says that, for every regular cardinal
\(\kappa\), we have a type \(\Type_\kappa\) whose \emph{elements} \(T :
\Type_\kappa\) are isomorphic to \(\kappa\)-small types.

We call a type like \(\Type_\kappa\), whose elements represent types, a
\emph{universe}.
%
\end{node}

\begin{node}
%
Abstracting away the administrative setup of regular cardinals, we will
assume that we have a countable set of symbols \(\ell, \ell_1, \dots\)
which we will call "levels", and we have a corresponding tower of
universes
\[
  \Type_{\ell} : \Type_{\ell_1} : \Type{\ell_2} : ...
\]
%
\end{node}

\begin{remark}
Our universes are not \emph{cumulative}, i.e., if we have a code \(A :
\Type_{\ell}\), then it is not automatically the case that the judgement
\(A : \Type_{\ell'}\) holds for \(\ell' > \ell\).
\end{remark}

\begin{node}
%
Additionally, our universes are all \emph{univalent}: there is a
canonical equivalence between the type of identifications in
\(\Type_\ell\) and the type of equivalences between \(\ell\)-small
types. \emph{Proving} this, rather than asserting it, is one of the
goals of cubical type theory.
%
\end{node}

\begin{remark}
The subobject classifier \(\Omega\) in a topos can also be seen as a
universe, with a very strict bound on cardinality: It classifies the
morphisms with \(\{*\}\)-small fibres.
\end{remark}

\begin{remark}
Our type theory does not have a "universe of all propositions," so that
every level \(\ell\) has an associated universe \(\Prop_\ell\) of
"\(\ell\)-small types with at most one element". Thus, the levels
\(\ell\) do not correspond to any particular assignment of
cardinalities.
\end{remark}
