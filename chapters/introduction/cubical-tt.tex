\begin{node}
%
\emph{Cubical type theory} is a type theory validating the univalence
principle, but, unlike axiomatic HoTT, it also enjoys canonicity. It is
a byproduct of the search for a constructive justification of the
univalence axiom, which resulted in the (many) models in \emph{cubical
sets}, including (but not limited to) those in
\cite{Bezem:2014,CCHM:2016,AFH:2017,Orton:2017,ABCFHL:2021}.
%
\end{node}

\begin{node}
%
By working backwards from the model, \cite{CCHM:2016} were able to
devise the syntax for a type theory in which univalence is provable, and
which enjoys canonicity. This system later became the basis for the
proof assistant Cubical Agda \cite{Vezzosi:2019}.
%
\end{node}

\begin{node}
%
The basic idea of cubical type theory is to represent the path types of
HoTT \emph{literally}, as functions out of an abstract interval \(\II\),
with specified endpoints. The main distinguishing feature between
cubical type theories is in the structure we equip \(\II\) with, and how
we integrate paths with the rest of the theory.
%
\end{node}

\begin{node}
%
This book is meant as an introduction to cubical type theory, assuming
basic knowledge of axiomatic HoTT. Thus, we will not delve too deeply
into the basic notions of type theory, but we will briefly recap the
necessary notions.

We will introduce cubical reasoning from first principles, and work
through establishing basic lemmas at a foundational level. Later, we
will see how cubical type theory finds concrete applications in
\emph{synthetic homotopy theory} and \emph{homotopy-coherent category
theory}.
%
\end{node}

\begin{node}
%
The material presented in this book is a refinement the author's
previous work, which was written accompanying a formalisation in Agda,
which is freely available online at \cite{1Lab:2022}.
%
\end{node}
