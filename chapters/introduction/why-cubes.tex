\begin{node}
%
Readers familiar with homotopy theory and other branches of higher
category theory may find the use of higher cubes, as opposed to higher
simplices, unusual.
%
\end{node}

\begin{node}
%
Our choice is motivated by the syntax of type theory. Terms and types
exist in \emph{contexts}, which are semantically represented by finite
products. Thus, type theory has a lightweight way of dealing with the
structure of iterated products: using names and contexts.

Thus, using a geometric shape for higher structures which plays well
with products essentially gives us a shortcut for representing the
geometry of higher paths --- just use contexts.
%
\end{node}

\begin{node}
%
A mathematical justification of this fact is given by the fact that
\(\square_\mathrm{cart}\), the cube category of \cite{ABCFHL:2021}, is
the free Cartesian monoidal category on a bipointed object.

Elaborating: If \(\mathcal{C}\) is a Cartesian monoidal category
equipped with a choice of object \(O\) and morphisms \(l, r : 1 \to O\),
then there exists an essentially unique finite-product-preserving
functor \(\square_\mathrm{cart} \to \mathcal{C}\) which sends the
interval to \(O\), sends the left endpoint to \(l\), and the right
endpoint to \(r\).
%
\end{node}

\begin{remark}
In contrast, the simplex category is well-suited for higher category
theory because it is a universal model of \emph{composition}: \(\Delta\)
is the free monoidal category equipped with a monoid object.
\end{remark}
