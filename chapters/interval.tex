\begin{node}
%
The star of the show in cubical type theory is the interval object
\(\II\). We shall devote this entire chapter to explaining its role and
behaviour in cubical type theory.
%
\end{node}

\begin{node}\label{node:interval-not-type}
We do not refer to the interval \(\II\) as an interval \emph{type}
because strictly speaking, it is not a type. Even so, we will use the
notation \(e : \II\) to indicate that \(e\) denotes an interval
expression. We will refer to variabels \(x : \II\) as \emph{dimensions}.
\end{node}

\begin{node}
%
The interval object \(\II\) is meant to represent, abstractly, the key
features of the real unit interval \([0,1] \subseteq \RR\) which make it
a good fit for defining \emph{paths}. Paths start and end, so the
interval has closed inhabitants (the \emph{endpoints}) called \(i0\) and
\(i1\).
%
\end{node}
